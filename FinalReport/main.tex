%%%%%%%%%%%%%%%%%%%%%%%%%%%%%%%%%%%%%%%%%
% Imperial (EEE/EIE) Final Year Project Report Template 
% LaTeX Template
% Version 1.0 (28/06/16)
% Version 1.1 (20/01/28) 
% Version 1.2 (02/06/19)
% Modified by Aufar Laksana into a Final Year Project Report template
% For academic use only
%%%%%%%%%%%%%%%%%%%%%%%%%%%%%%%%%%%%%%%%%
%----------------------------------------------------------------------------------------
%	PACKAGES AND OTHER DOCUMENT CONFIGURATIONS
%----------------------------------------------------------------------------------------

\documentclass[11pt,a4paper]{report}
\usepackage[english]{babel}
\usepackage[utf8x]{inputenc}
\usepackage{amsmath}
\usepackage{amsfonts}
\usepackage{graphicx}
\usepackage{fancyhdr}
\usepackage[colorinlistoftodos]{todonotes}
\usepackage[toc,page]{appendix}
\usepackage[procnames]{listings}
\usepackage[page]{totalcount}
\usepackage{color}
\usepackage[bottom=10em]{geometry}
\usepackage{caption}
\usepackage{subcaption}
\usepackage{float}
\usepackage[multiple, bottom]{footmisc}
\usepackage{diagbox}
\usepackage{gensymb}
\usepackage{mathpazo}

\usepackage{wrapfig}
\usepackage{lscape}
\usepackage{rotating}
\usepackage{epstopdf}


\usepackage{natbib} 

\usepackage{algorithm2e}

\definecolor{mygreen}{rgb}{0,0.6,0}
\definecolor{mygray}{rgb}{0.5,0.5,0.5}
\definecolor{mymauve}{rgb}{0.58,0,0.82}
\definecolor{mylilas}{RGB}{170,55,241}

\definecolor{deepblue}{rgb}{0,0,0.5}
\definecolor{deepred}{rgb}{0.6,0,0}
\definecolor{deepgreen}{rgb}{0,0.5,0}

\definecolor{keywords}{RGB}{255,0,90}
\definecolor{comments}{RGB}{0,0,113}
\definecolor{red}{RGB}{160,0,0}
\definecolor{green}{RGB}{0,150,0}

\pagestyle{fancy}
\fancyhf{}
\lhead{\leftmark}
\rhead{FINAL REPORT}
\rfoot{\thepage\ / \totalpages}

% \geometry{headheight=15pt}
%\geometry{footskip=1.2in}
% \geometry{textheight=694pt}
% \geometry{textwidth=400pt}

\lstset{ %
	basicstyle=\small,
	backgroundcolor=\color{white},   % choose the background color; you must add \usepackage{color} or \usepackage{xcolor}; should come as last argument
	breaklines=true,                 % sets automatic line breaking
	captionpos=b,                    % sets the caption-position to bottom
	commentstyle=\color{mygreen}\ttfamily\small,    % comment style
	escapeinside={\%*}{*)},          % if you want to add LaTeX within your code
	extendedchars=true,              % lets you use non-ASCII characters; for 8-bits encodings only, does not work with UTF-8
	frame=shadowbox,	                   % adds a frame around the code
	rulesepcolor=\color{teal},
	keepspaces=true,                 % keeps spaces in text, useful for keeping indentation of code (possibly needs columns=flexible)
	keywordstyle=\color{blue},       % keyword style
	language=C,                 % the language of the code
	morekeywords={*,...},            % if you want to add more keywords to the set
	numbers=left,                    % where to put the line-numbers; possible values are (none, left, right)
	rulecolor=\color{black},         % if not set, the frame-color may be changed on line-breaks within not-black text (e.g. comments (green here))
	showspaces=false,                % show spaces everywhere adding particular underscores; it overrides 'showstringspaces'
	showstringspaces=false,          % underline spaces within strings only
	showtabs=false,                  % show tabs within strings adding particular underscores
	stringstyle=\color{mymauve},     % string literal style
	tabsize=2,	                   % sets default tabsize to 2 spaces
}

\lstset{language=Python, 
	basicstyle=\ttfamily\footnotesize, 
	keywordstyle=\color{deepblue},
	commentstyle=\color{deepgreen},
	stringstyle=\color{deepred},
	showstringspaces=false,
%	identifierstyle=\color{},
	procnamekeys={def,class, for, if}
	tabsize=3,
	showstringspaces=false}

%\setmonofont{Consolas} %to be used with XeLaTeX or LuaLaTeX
\definecolor{bluekeywords}{rgb}{0,0,1}
\definecolor{greencomments}{rgb}{0,0.5,0}
\definecolor{redstrings}{rgb}{0.64,0.08,0.08}
\definecolor{xmlcomments}{rgb}{0.5,0.5,0.5}
\definecolor{types}{rgb}{0.17,0.57,0.68}

\usepackage{listings}
\lstset{language=[Sharp]C,
captionpos=b,
%numbers=left, %Nummerierung
%numberstyle=\tiny, % kleine Zeilennummern
frame=lines, % Oberhalb und unterhalb des Listings ist eine Linie
showspaces=false,
showtabs=false,
breaklines=true,
showstringspaces=false,
breakatwhitespace=true,
escapeinside={(*@}{@*)},
commentstyle=\color{greencomments},
morekeywords={partial, var, value, get, set},
keywordstyle=\color{bluekeywords},
stringstyle=\color{redstrings},
basicstyle=\ttfamily\small,
}


\lstdefinelanguage{Mymatlab}{
	language=Matlab,%
	%basicstyle=\color{red},
	basicstyle=\ttfamily\footnotesize,
	breaklines=true,%
	morekeywords={matlab2tikz},
	keywordstyle=\color{blue},%
	morekeywords=[2]{1}, keywordstyle=[2]{\color{black}},
	identifierstyle=\color{black},%
	stringstyle=\color{mylilas},
	commentstyle=\color{mygreen},%
	showstringspaces=false,%without this there will be a symbol in the places where there is a space
	numbers=left,%
	numberstyle={\tiny \color{black}},% size of the numbers
	numbersep=9pt, % this defines how far the numbers are from the text
	emph=[1]{for,end,break},emphstyle=[1]\color{red}, %some words to emphasise
	%emph=[2]{word1,word2}, emphstyle=[2]{style},    
}
\lstdefinelanguage{TI}{
	sensitive = true,
	keywords={MVC,MVK,MVKLH,LDDW,LDW,NOP,STW,ZERO,LDDW,MPYDP,ADDDP,SUB,B},
	otherkeywords={% Operators
		>, <, ==
	},
	keywords = [2]{_circ_FIR_DP,loop,lend},
	keywordstyle=\color{blue},
	keywordstyle=[2]\color{purple},% for example
	numbers=left,
	numberstyle=\scriptsize,
	stepnumber=1,
	numbersep=8pt,
	showstringspaces=false,
	breaklines=true,
	frame=shadowbox,	                   % adds a frame around the code
	rulesepcolor=\color{teal},
	comment=[l]{;},
	morecomment=[s]{/*}{*/},
	commentstyle=\color{mygreen}\ttfamily\small,
	stringstyle=\color{red}\ttfamily,
	morestring=[b]',
	morestring=[b]"
}


\newcommand{\code}[1]{\texttt{#1}}

\newcommand{\HRule}{\rule{\linewidth}{0.5mm}} 

\begin{document}

% Title Page
\begin{titlepage}
	% \newgeometry{top=25mm,bottom=25mm,left=38mm,right=32mm}
	\setlength{\parindent}{0pt}
	\setlength{\parskip}{0pt}
	% \fontfamily{phv}\selectfont
	
	{
		\Large
		\raggedright
		Imperial College London\\[17pt]
		Department of Electrical and Electronic Engineering\\[17pt]
		Final Year Project Report 2019\\[17pt]
		
	}
	
	\rule{\columnwidth}{3pt}
	\vfill
	\centering
	\vfill
	\setlength{\tabcolsep}{0pt}
	
	\begin{tabular}{p{40mm}p{\dimexpr\columnwidth-40mm}}
		Project Title: & \textbf{Augmented Reality For Human Robotic Interaction} \\[12pt]
		Student: & \textbf{Aufar P. Laksana} \\[12pt]
		CID: & \textbf{01093575} \\[12pt]
		Course: & \textbf{EIE4} \\[12pt]
		Project Supervisor: & \textbf{Dr Yiannis Demiris} \\[12pt]
		Second Marker: & \textbf{Dr Tae-Kyun Kim} \\
	\end{tabular}
\end{titlepage}

\addvspace{6em}

\renewcommand{\abstractname}{\LARGE Abstract}

\begin{abstract}
Powered wheelchairs are becoming increasingly commonplace in the modern world. However, a significant issue faced by powered wheelchair users (PWUs) is navigating the device in crowded areas. Controlling the powered wheelchair in crowded areas requires increased concentration from the PWU, as people in crowds often move unpredictably, or are hidden from view due to standing behind another person or object.

\paragraph{}This project utilises computer vision techniques to predict the direction of travel of individuals in crowds and implements an augmented reality system using the Microsoft Hololens that helps the PWU by displaying visual aids that indicate the motion of people. The system further helps the user by warning them of potential collisions, allowing the PWU to make better navigation decisions. The project also explores the use of the system as a method of reactive control of the wheelchair, preventing collisions by stopping the powered wheelchair should the PWU not attempt to avoid a collision with an individual crossing their path.

\paragraph{}The result of this project is an augmented reality system prototype that can be used to aid navigation tasks for PWUs. Our testing shows that the implemented system can detect collision risks in a controlled environment and react appropriately using only a visual camera input. This report also discusses the disadvantages of relying on only one form of sensor input for obstacle avoidance and the limitations of using the Hololens as a head mounted aid.
\end{abstract}


\newpage

\paragraph{Acknowledgements} This is for a few people

\paragraph{}To Professor Yiannis Demiris, thank you for giving me the freedom to shape this project as I liked. You allowed me to explore fields that interest me and gave me the opportunity to work with the Hololens and other devices that would normally be out of reach. This project has re-ignited my passion for engineering, problem solving and making things work.

\paragraph{}To Rodrigo, Mark and Yong, thank you for always being willing to answer my questions, guiding me through the maze that is Unity, Hololens and ROS development. Without your help and support, this project would not have been possible.

\paragraph{}To my friends and colleagues on the 5\textsuperscript{th} floor office, Marek, Shrey, Tom, Zihan, Ian \& Abdullah, thank you for sitting there with me as we worked on our seperate final year projects. We have laughed, we have stressed but ultimately we have all become friends. I will miss coming into the office everyday and talking with you guys, and I wish you all the best of luck in whatever you choose to pursue.

\paragraph{}To my Mum and Dad, Dian and Bawa, thank you for all the sacrifices you have made to give me the oppurtunity to study at this prestigious university. I have never admitted it, but Dad, I have always seen you as a rolemodel to beat, and I have tried and will continue to try to surpass the expectations you have set. I will never be able to repay you for everything you have done for me, but I will continue to strive to make you both proud.

\paragraph{}To my younger brother Ammar, who is about to start university, thank you for being my brother. You have always supported me and despite our quarrels and disagreements, I could not ask for a better sibling. I hope you do well in university, and that you enjoy your time there.



\tableofcontents

\newpage

\chapter{Introduction}

\section{Introduction}
This report was written as part of the Final Year Project for the MEng Electronic \& Information Engineering course. The project was supervised by Professor Yiannis Demiris at the Imperial College London.

\section{Motivation \& Objectives}
Powered wheelchairs are becoming increasingly commonplace in the modern world. As technology advances, the devices have become more accessible and smarter, with schemes such as assistive control which help the user in navigation. Within the Personal Robotics Lab (PRL) at Imperial College London, one of the major research topics revolves around powered wheelchairs, and how the user experience can be enhanced. Previous work in the lab has utilized augmented reality headsets to enhance the user experience, by displaying visual aids to the powered wheelchair user (PWU) \cite{Zolotas2018, Chacon-Quesada} that help the PWU understand the internal state of the device, such as the current trajectory or planned path. However, despite the advances in smart wheelchairs and user enhancement techniques, a study showed that one of the concerns of PWUs is navigating in crowded spaces \cite{Kairy2014}. This concern is due to the unpredictability of people in crowds and their tendency to change direction rapidly. 

\paragraph{}As such, this project explores the development of an augmented reality experience using the Microsoft Hololens as an aid for PWUs. The goal of the system would be to detect people in the surroundings and display the corresponding visual aids to the PWU to indicate potential collisions. To further alleviate the concerns of PWUs, the system should also be able to avoid accidents by reacting appropriately when the PWU is unaware of a potential collision.

\section{Challenges}
This project highlights the problems associated with the current version of the Microsoft Hololens. First and foremost is the lack of a built-in library for accessing the raw frames of the front facing camera, as well as a library for video streaming to partner devices. Secondly, the lack of a UWP implementation of the Robotic Operating System meant we were forced to modify a third-party library to be able to communicate between the Hololens running a Unity application and ROS nodes on the partner PC. Finally, we observed the limitations of the spatial mapping capabilities of the Hololens, and how they affect hologram stability and placement accuracy. 

\section{Contributions}
By developing this augmented reality system, we have contributed:

\begin{itemize}
	\item A Unity Engine application that can stream the front-facing camera of the Hololens across the network.
	\item A ROS package for the Darknet Neural Network framework.
	\item A trained object detector for detecting partially occluded people in crowds.
	\item A ROS wrapper for the OpenPose body pose estimation network.
	\item A Unity application that can visualize the position of detected persons by rendering holograms at the detected positions.
	\item A reactive control system for powered wheelchairs that avoid collision by manipulating the velocity of the device when it detects a collision risk.
\end{itemize}

\section{Report Structure}
This report begins with the background research that was done before embarking on the development of the system. We explore different computer vision techniques for people detection and tracking, as well as robotic mapping techniques for objects in the surroundings. We then discuss the requirements of the system we implemented, before describing the system from a high-level perspective, as well as why certain design decisions were taken.

\paragraph{}The implementation section of this report covers the steps taken to develop the final system, from training the object detector to the reactive control system for the powered wheelchair. We then describe the testing procedure that allows us to analyse the performance of our product. Finally, we discuss the performance and limitations of the system we implemented before highlighting how the overall system could be improved in future work.
\chapter{Background}
This project is focused on computer vision for detecting and tracking humans in the surroundings, estimating their trajectories and distance from the PWU, the reactive control systems that prevent collisions with the detected objects as well as the augmented reality display to provide visual cues to the PWU.

\section{Human Detection}
Human detection is a subset of the classic computer vision problem of object detection. In order to develop an augmented reality system that will help PWUs to navigate in public spaces, it is essential for the system to be able to discern humans from the surroundings.

\subsection{Definition of Requirements}
 The problem arises in crowded areas, whereby individuals are occluded by other people or objects in front of them, leaving only certain body parts visible. As such, we began our research with the problem of being able to detect people in images where identifiying parts of the body are not always visible.

\subsection{Review of Existing Methodologies}
A related field of research is that of people counting and human detection in visual surveillance in public areas. Where the problem differs is that surveillance benefits from being able to rely on cameras with a good view of the crowd from above, whereas for a PWU, the camera will not have as high of a vantage point, making detecting every single individual in a crowd impossible. 

Despite the disadvantage, similar techniques can be used to detect humans in video. Most methods can be classified into two categories \cite{Hou2010}. The first technique, foreground detection, attempts to model the background of an image and then detect the changes that occur between frames. The second category involves exhaustively searching the image with a scanning window, and deciding if each window can be classified into a human shape.

\begin{figure}[ht]
	\begin{subfigure}[b]{.5\textwidth}
		\centering
		\includegraphics[width=.9\linewidth]{img/chapter2_background/robustBackgroundSubtraction.png}
		\caption{Foreground Detection \cite{Zeng2017}}
	\end{subfigure}%
	\hspace{\fill} 
	\begin{subfigure}[b]{.5\textwidth}
		\centering
		\includegraphics[width=.675\linewidth]{img/chapter2_background/yoloBBBoxes.png}
		\caption{Scanning Window \cite{Redmon}}
	\end{subfigure}
	
	\begin{center}
		\caption{Comparison of Foreground Detection and Scanning Windows}
		\label{fig:foregroundVsScanning}
	\end{center}
\end{figure}

\subsubsection{Foreground Detection}
Background subtraction is a widely used approach for detecting moving objects \cite{Piccardi2004}. A temporal average filter can be used to find the median of all the pixels in an image to form a reference image. Frames with moving objects can then be compared pixelwise to the reference, and a threshold set to determines if the pixel is part of the background or foreground. People counting and human detection can then be achieved by segmenting the foreground image into individuals.

\paragraph{} However, this technique relies on a static camera in a well placed location. This brings up several reasons as to why this method would not be suitable for this project. Firstly, the camera available is part of a head-mounted augmented reality device. The wearer has the ability to move the camera in 6 degrees of freedom. Secondly, the wearer will also be navigating a powered wheelchair. As a result, the background is constantly changing, and the reference image would require constant recomputation before human detection can even begin.

\subsubsection{Scanning Windows}
Due to the ever-changing surroundings of a mobile robot, a better approach for object detection is to exhaustively search an image using scanning windows and determining if an object was detected in each window. However, it must be noted that this method is computationally expensive. In order to achieve real-time detection on a mobile robot, the use of a graphics processing unit (GPU) should be considered \cite{Hirabayashi}.

\subsubsection{Classical Object Detection}

\paragraph{Haar Cascades}
Haar cascades classifies images based on the value of simple features \cite{Viola2001}, which are variants of the difference between the sum of pixel values in rectangular regions. An intermediate representation of the original image is used to rapidly compute a small set of representative rectangular features.

\paragraph{}A cascade of classifiers is then used to determine if the region is detected as a human. The detection process is that of a degenerate decision tree, where a positive result in the first cascade will trigerr an evaluation in the second, more successful classifier. As such, the initial classifier can eliminate a large number of negative examples with very little processing. After several stages, the number of sub-windows has been reduced radically

\paragraph{Histograms of Oriented Gradients}
The method proposed is implemented by dividing the image window into small spatial regions and calculating a local 1-D histogram of gradient directions for all the pixels in the region. The combined local histograms form the overall feature representation of the image.

\paragraph{}The detection window is tiled with the Histogram of Oriented Gradient (HOG) descriptors. In the original paper \cite{Dalal2005}, these feature vectors were then used in a conventional SVM based window classifier  to give human detections.

\subsubsection{Deep Learning Object Detection}
Modern approaches for human detection largely depend on Deep Convolutional Neural Networks (CNN). The approach provides the best in class performance, as well as scaling effectively with more data. An added advantage of using CNN based object detection systems for this project is that they are also capable of detecting multiple classes of objects.

\paragraph{}An issue with CNN approaches is that the methods are trying to draw bounding boxes around objects of interest in images. However, we do not know the number of objects in the image beforehand. As such, to be completely sure every object has been detected, a naive solution is to take a huge number of regions and attempt to classify all the objects in the region, a computationaly expensive process.

\begin{figure}[ht]
	\centering
	\includegraphics[width=.9\linewidth]{img/chapter2_background/rcnn.png}
	\caption{R-CNN Approach}
\end{figure}

\paragraph{R-CNN} The R-CNN method uses a selective search to extract 2000 regions from an image \cite{Girshick2014}. The regions are selected by generating a large number of candidate regions and using a greedy algorithm to recursively combine similar regions into larger ones. The regions are then fed into a CNN that acts as a feature extractor and the output dense layer consists of the features extracted from the image, which are then fed into an SVM to classify the presence of objects in the region.

\paragraph{}The major disadvantage to this approach is the amount of time required to train the network. Each training image has to be classied once for each of the 2000 region proposals. Furthermore, the selective search algorithm is a fixed algorithm (no learning is done), and as such, could lead to generation of bad candidate region proposals.

\paragraph{YOLO} Whereas R-CNN uses regions to localize the object within an image, You Only Look Once (YOLO) looks at the image as a whole and uses a single CNN to predict the bounding box and the class probabilities \cite{Redmon}. By looking at the image as a whole, the network can use features from the entire image to predict each bounding box.

\begin{figure}[ht]
	\centering
	\includegraphics[width=.6\linewidth]{img/chapter2_background/yoloApproach.png}
	\caption{YOLO Approach}
\end{figure}

The model divides the image into an $S \times S$ grid, and for each cell, predicts a number of bounding boxes, the confidence for those boxes and the class probabilities.

\subsection{Comments}
As seen from the research, we can clearly see that there are many ways to solve the human detection problem. The classical approaches, although computationally efficient, are siginficiantly outperformed by the deep learning approaches. For a mobile robot in a public area, we want to be able to detect almost all humans in the surroundings to better inform the PWU.

\paragraph{}However, the major disadvantage of the deep learning approach is the time taken to train the network, as well as the requirement of a GPU to achieve real-time performance. These issues will be addressed in a later section of the report.

\section{Object Tracking}
% Object tracking
% Track objects across frame to get a sense of direction

\section{Head Pose Estimation}

\section{Body Pose Estimation}
% Human body pose estimation
% The pose of the shoulders and body can be used to determine the direction

\section{SLAM}

\section{Augmented Reality Headsets}
\chapter{Requirements Capture}

\section{Project Deliverable}
The objective of this project is to develop an augmented reality system that can be used by powered wheelchair users (PWUs) to assist them in navigating their powered wheelchairs in public spaces with many people walking in the surroundings. The system should be able to detect the presence of individuals and infer their position relative to the PWU, and by extensions, estimate their direction of travel.

\paragraph{}We propose a system that uses the Microsoft Hololens augmented reality headset as the primary input and visualization tool. The PWU would wear the headset as they operate the powered wheelchair, allowing the system to create visualizations of potential obstacles and collisions. Furthermore, the system would also encompass the reactive control aspect of the powered wheelchair. Should an individual be detected as walking in the wheelchairs trajectory, the system will send control signals to the powered wheelchair to slow down or completely stop depending on how far the target is from the wheelchair.

\paragraph{} By definition of the requirements, we can divide the project into three parts: Human Detection and Direction, Obstacle Mapping \& Visualization, and finally, Reactive Control.

\section{Human Detection and Direction} \label{sec:reqHDD}
The requirements of the Human Detection and Direction (HDD) system is to be able to use a video stream of the surroundings to determine the position and direction of people. The Hololens has a built-in camera that can be used to take photos of the surroundings of the user \cite{Chacon-Quesada}. We will leverage this ability to create a video stream.

\paragraph{}The actual HDD system is implemented on another computer with access to a GPU. We utilize the GPU to be able to do real-time object detection and pose estimation of detected individuals. The system should be able to infer the direction individuals are walking in, as well as their real-world positions relative to the PWU.

\paragraph{Features} 
\begin{itemize}
    \item Creating a live video stream using the front-facing camera.
    \item Streaming the live video to accompanying computer.
    \item Object detector trained on humans/pedestrians.
    \item Object tracker to track detected humans, and determine their direction.
    \item Body/Head pose estimator to determine the direction of travel.
    \item Stream detections back to the Hololens for visualization.
\end{itemize}

\section{Obstacle Mapping \& Visualization }
This project utilizes the Microsoft Hololens as a visualization and spatial mapping tool. The HDD system will output its detections and directions to the Hololens, which is used to create visualizations that will help the PWU. Examples of the visualizations include arrows that indicate the direction of movement, as well as alerting the user to potential collisions.

\paragraph{Features}
\begin{itemize}
    \item Receiving detection/direction data from HDD system.
    \item Utilize Camera to World transforms of the Hololens Camera to get World coordinates of people.
    \item Create holographic visualizations to help PWU understand the direction people are walking in.
    \item Create a map of obstacles for Reactive Control.
\end{itemize}
 
\section{Reactive Control} \label{sec:reactive}
The powered wheelchair (ARTA) available in the Personal Robotics Lab (PRL) can be manually operated using a joystick. The goal of the project is for the PWU to be able to wear the Hololens as an aid for navigation in public spaces. As such, it would be beneficial for the PWU if the wheelchair could reactively control the device to prevent collisions with detected objects.

\paragraph{Features}
\begin{itemize}
    \item Receiving object detections in front of the wheelchair.
    \item Prevent wheelchairs from driving into objects.
\end{itemize}

\chapter{Analysis and Design}
This chapter gives an overview of the overall system and explains the design choices made. Throughout the project, we explored various methods to implement a real-time augmented reality system for PWUs operating a wheelchair. Naturally, the structure and goals of the project have developed since the interim report, and we review the differences between the initial goals and final product .

\section{Design Overview}
As stated in the requirements, this project consists of three major components:
\begin{itemize}
	\item Human Detection and Direction (HDD)
	\item Object Mapping and Visualization (OMV)
	\item Reactive Control on ARTA
\end{itemize}

\begin{figure}[ht]
	\centering
	\includegraphics[width=1.0\linewidth]{img/chapter4_analysis/simpleSystemDiagram.png}
	\caption{Simplified high level system diagram}
	\label{fig:simplifiedHL}
\end{figure}

From a very high level view, we can map these requirements to the respective devices they will be operating on. The HDD system takes implements the object detection and human direction inference, while the Hololens is responsible for utilizing the spatial mapping to obtain world positions of the detections, as well as visualizing the detections. The powered wheelchair (ARTA), has manual input that is overridden by the reactive control system that is dependent upon the detections and mappings. The diagram in Figure. \ref{fig:simplifiedHL} shows an overview of the system, and shows that the Hololens acts as an intermediary between ARTA and the HDD.

\subsection{Hardware}

\begin{table}[ht]
	\centering
	\begin{tabular}{l|l|l|l|}
		\cline{2-4}
		& \multicolumn{1}{c|}{\textbf{ARTA}}                                                 & \multicolumn{1}{c|}{\textbf{Hololens}} & \multicolumn{1}{c|}{\textbf{HDD}}                              \\ \hline
		\multicolumn{1}{|l|}{\textbf{Hardware}}         & \begin{tabular}[c]{@{}l@{}}Powered Wheelchair \\ controlled by Laptop\end{tabular} & Hololens                               & \begin{tabular}[c]{@{}l@{}}Desktop PC \\ with GPU\end{tabular} \\ \hline
		\multicolumn{1}{|l|}{\textbf{Operating System}} & Ubuntu 16.04                                                                       & UWP                                    & Ubuntu 16.04                                                   \\ \hline
	\end{tabular}
	\caption{Hardware description of system}
	\label{tab:hardware}
\end{table}

Table. \ref{tab:hardware} summarises the hardware overall system is implemented on. The powered wheelchair, ARTA, is controlled by a laptop, which is responsible for the wheelchair speed, wheel rotations, navigation and localisation. The Hololens is a self contained augmented reality headset, running the Universal Windows Platform (UWP) operating system. Finally, the Human Detection \& Direction system is implemented on a desktop computer with a GTX 1050Ti GPU, allowing it to run real time object detectors.

\subsection{System 
	Communication}

\subsubsection{Robotic Operating System} Since the project spans multiple operating systems, we have chosen to utilize the Robotic Operating System (ROS) as a means of communication between the devices. In ROS, a \textit{node} is defined as a process that performs a computation. A node can be made up of smaller nodes that perform specific computations that serve the needs of the parent node. 

\begin{figure}[ht]
	\centering
	\includegraphics[width=1.0\linewidth]{img/chapter4_analysis/detailedSystemDiagram.png}
	\caption{System diagram detailing individual components}
	\label{fig:detailedHL}
\end{figure}

We can think of the three major systems as large ROS nodes that consists of smaller nodes that run individiual tasks, such as creating the camera stream, or detecting objects. We visualize the breakdown of the system into nodes in Figure. \ref{fig:detailedHL}.

\paragraph{ROS Topics} Nodes in ROS communicate with one another by publishing data in the form of \textit{messages} which get broadcasted over a \textit{topic}. Nodes can choose what data they receive by subscribing to topics. This method allows for nodes running on different devices to communicate with each other, regardless of the operating system. The nodes are unaware that the data it receives is published from a node running on a seperate computer, making ROS a perfect choice for communication in this design.

\section{Human Detection \& Direction System}
The HDD system is reponsible for detecting and predicting the directions of people in the surroundings of the wheelchair. By taking visual inputs in the form of images from the Hololens, we run an object detector trained on a dataset of pedestrians to detect people and heads. The bounding boxes produced by the object detector are fed as inputs to an object tracker and a body pose estimator. We use the results of these two nodes to infer the direction a detected person is moving in, and publish the results back to the Hololens.

\begin{figure}[ht]
	\centering
	\includegraphics[width=1.0\linewidth]{img/chapter4_analysis/hddSystemDiagram.png}
	\caption{Individual nodes in HDD System}
	\label{fig:detailedHDD}
\end{figure}

We present an overall view of the HDD System, covering the purpose and design of individual components. We also propose the reasoning behind certain design choices, which we cover in more depth later in this report.

\subsection{YOLO Object Detector}
Object detectors often form the input to an object tracker or pose estimation system \cite{Bewley2016, Jin2017}. In the case of top-down body pose estimation methods, detections can be the first point of failure \cite{Insafutdinov}. As such, the accuracy of the chosen object detector must be considered, together with the choice of using a pre-trained model or training on a more relevant dataset. Finally, we must also consider the use-case of the detector, which must be able to operate in real-time and detect moving objects as they pass by.

\subsubsection{Choice of Detector}
As commented on in Section \ref{sec:detector}, modern deep learning techniques outperform classical object detectors in accuracy, but are limited by the requirement of a GPU to perform in real-time. Since the Hololens does not have built in support to run object detection networks, Microsoft provides the Azure Cognitive Services API to allow developers to query their system for object detections. The limitation is that this service is not free, and abstracts away the implementation of an object detector. Furthermore, one of the personal goals for this project was to learn more about CNNs in computer vision.

\paragraph{} Taking this into account, we compared several deep learning architectures for object detection. Previous work done in the PRL used Facebook AI Research's (FAIR) Detectron to detect objects \cite{Chacon-Quesada, Detectron2018}. Further discussions with members of the Imperial Computer Vision \& Learning Lab suggested the use of the YOLO object detector \cite{Redmon}, due to its speed and having a lightweight implementation that could be run on lower end GPUS at relatively high frame rates. This prompted the design decision to use the Darknet framework to run the \textbf{YoloV3-Tiny} architecture as the object detection method of choice for this project \cite{Redmon2018}.

\subsubsection{Pre-trained model vs Training}



\subsection{YACT: Yet Another Crowd Tracker}

\subsubsection{Object Tracking}

\subsubsection{Body Pose Estimation}

\subsubsection{Head Pose Estimation}

\section{Hololens}
\subsection{Breakdown}

\section{ARTA}
\subsection{Breakdown}



\chapter{Implementation}

\paragraph{Algorithm} Given an image and corresponding bounding box co-ordinates, we can determine if an object is already tracked and update its new position as follows:

\begin{algorithm}[ht] %https://en.wikibooks.org/wiki/LaTeX/Algorithms#An_example_from_the_manual
	\KwData{Test}
\end{algorithm}
\chapter{Testing \& Results}
This chapter details the testing that was carried out to assess the performance of different parts of the system, such as the human detection and direction, as well as the powered wheelchair and Hololens system as a whole. We outline the test setups used to evaluate the performance of the systems, as well as the results of the tests and what they imply about the implemented system.

\section{Human Detection and Distance}
This section is concerned with testing the Human Detection \& Direction system and the Hologram GameObject placement. To ensure the system will be able to detect real people moving around in the surroundings, it is necessary to test the system detecting people at different distances and the directions they are facing. We also want to test the accuracy of the spatial mapping system in terms of correcting the real world positions of the GameObjects representing the detected people. We utilize the Microsoft Hololens and the HDD system for this test.

\subsection{Test System Description} \label{sec:hddSys}
As explained in the Implementation chapter of this report, the front facing camera uses ROS topics to stream video frames to a partner computer the HDD system is implemented on. The HDD processes the frames and detects people and determines whether they are facing the PWU or not. The bounding box of the detections and directions are sent back to the Unity application. Initially, the application converts the pixel co-ordinates of the detections to corresponding world co-ordinates. We then use the spatial mapping and ray casting capabilities of the Hololens to correct the world position distances of the holograms representing the detected people.

\subsection{Test Setup} \label{sec:testSetup}
We setup a testing ground in the ICRS Lab on the 5th floor of the EEE building. We marked out points at 1 meter intervals which indicate where the target people should stand. We asked the person wearing the Hololens to sit down in a chair to emulate the position and height a PWU would be at when sitting in the wheelchair. Figure \ref{fig:hddTestSetup} shows the experimental setup in the lab. For this step, all subjects were stationary, except when the target person moves between the markings.

\begin{figure}[ht]
	\centering
	\includegraphics[width=1.0\linewidth]{img/chapter6_test/hddTestSetup.png}
	\caption{The experimental setup used to test the HDD and Hololens spatial mapping accuracy. We asked the people acting as targets to stand at 1 meter intervals while a person sitting down wearing the Hololens looked at them using the front facing camera.}
	\label{fig:hddTestSetup}
\end{figure}
 
We modified the Hololens Unity application to display holograms showing the distance between the Hololens and the target person in meters. We asked the person wearing the Hololens to call out the value the hologram displayed as the target person stood at different markings. To verify the readings, we recorded the Hololens display using the Windows Device Portal and watched them over after the test. Recording the hololens view also allowed us to see how the distance reading change as the target person moves.
 
\subsection{Test Procedure}
We asked three different people to sit in the chair and wear the Hololens. Before commencing the test, we allowed the people to learn how to wear the Hololens properly, ensuring the Hololens was correctly placed on the bridge of the users nose. We also allowed them to do several test detections to familiarise them with how the distance holograms would be rendered. This was done so that each participant would be able to detect the target person and call out the distance displayed by the hologram.

\paragraph{} The selected people all had different amounts of experience with the Hololens. One user was a complete novice, putting the Hololens on for the first time to participate in the test. One was an intermediate, having worn the Hololens a few times during the development of this project and the last person was more experienced with the Hololens and how the HDD system would detect the target people.

\paragraph{} We asked the target person to face forward at each marker, before turning their back and facing away from the Hololens user. This was done to check if there were any differences between the placement accuracy of the GameObjects for different orientations of the person.

\subsection{Results}
We show the results of the experiment in Figure \ref{fig:hddResults}, which compares the accuracy of hologram placement after spatial mapping and ray casting with the true world position of the detected person. Each test subject was given several seconds to adjust their head positions to ensure a detection was achieved. Despite this, none of the three test subjects were able to get holographic distances for a target at $6$m or further. This may indicate that the system was unable to detect a person at that distance, but what is more likely  is that the spatial mapping of the Hololens was unable to detect a surface that far away to render the distance hologram on. By looking at the debug log messages in Visual Studio 2017, we verified that there were indeed detections at $6$m, but they returned inaccurate values distance values.

\begin{figure}[ht]
	\centering
	\includegraphics[width=1.0\linewidth]{img/chapter6_test/hddtestresults.png}
	\caption{The graph shows that there is no significant difference between the orientation of the target. We note the decrease in accuracy past $4$m, and the lack of holograms at $6$m.}
	\label{fig:hddResults}
\end{figure}

The data shows that there is no significant difference in hologram placement distance for target persons facing towards or away from the Hololens. However, we note that the hologram placement accuracy decreases the further away the target person is from the Hololens. This is further shown from the Hololens view recordings in Figure \ref{fig:marek}, whereby we can see that the placement of the hologram is not on the target person, but a surface just behind the target.

\begin{figure}[ht]
	\begin{subfigure}[b]{.32\textwidth}
		\centering
		\includegraphics[width=1.0\linewidth]{img/chapter6_test/marek.png}
		\caption{Subject at 2m}
	\end{subfigure}%
	\hspace{\fill} 
	\begin{subfigure}[b]{.32\textwidth}
		\centering
		\includegraphics[width=1.0\linewidth]{img/chapter6_test/marek1.png}
		\caption{Subject at 3m}
	\end{subfigure}
	\hspace{\fill} 
	\begin{subfigure}[b]{.32\textwidth}
		\centering
		\includegraphics[width=1.0\linewidth]{img/chapter6_test/marek2.png}
		\caption{Subject at 6m}
	\end{subfigure}
	\vspace{-1\baselineskip}
	\begin{center}
		\caption{Target person at different distances viewed through the front-camera of the Hololens. We note that the accuracy of the hologram placement decreases as the target is further away.}
		\label{fig:marek}
	\end{center}
	\vspace{-2\baselineskip}
\end{figure}

\paragraph{} After the test, we asked the test subjects on qualitative feedback on how easy it it is to view a hologram. As expected, the novice Hololens user had difficulty seeing holograms due to the limited FOV of the device. The novice stated that the holograms were being rendered too high up, as they could only see the bottom of the hologram, as the rest of it was slightly out of view unless they tilted their head upwards. The intermediate user had a similar experience, but only had difficulty in seeing the holograms when the target person was closer than 2 meters.

\subsection{Discussion} \label{sec:distanceDiscussion}
From our results, we note that the test subjects were unable to see a hologram when the target person was more than 5m away, which indicates that there is a limitation with the spatial mapping of the Hololens. We suspect that this is due to a surface mesh not being rendered for the detected person, since the spatial mapping class was unable to discern between the person and the background at that distance. Secondly, we bring up the issue of hologram placement accuracy compared to the real world position of the target person. This issue is visualized by the graph in Figure \ref{fig:hddResults}, which shows a drop in accuracy for distances beyond 4 meters. Furthermore, Figure \ref{fig:marek}.b shows the hologram being rendered on a surface behind the target person.

\paragraph{} We believe that these issues are caused by ray casting through an incorrect world co-ordinate. Firstly, The frames captured by the front facing camera have to be compressed and streamed to a partner PC. The frames are then processed by the HDD system before being streamed back across the network. The culmination of the time spent being processed and streamed means that there is a slight delay between the detection of the person and their current position as interpreted by the Hololens. As such, when the Hololens un-projects the image co-ordinate of the detection to obtain a world co-ordinate, the placement of the initial GameObject is slightly incorrect. The system uses the ray casting function to pass a ray through the current position and expects the spatial mapping to detect a surface representing the detected person to position the hologram on. However, due to the slight delay, the ray continues to travel until it hits the background mesh instead of the person. This would explain why holograms are correctly placed on detections which are nearer, since the person is larger in the frame, and as such, has more surface mesh area for the ray cast to hit.

\section{Gazebo Simulation} \label{sec:gazeboSimTest}
Before testing the system using ARTA in the real world, we used the robot simulation tool Gazebo to test how the reactive control system would manipulate the ARTA control signals in response to person detections at different distances. The purpose of this simulation is to test the communication between ARTA and the Hololens, and how the reactive control system manipulates the velocity of the wheelchair as it approaches a detected person. We also want to test if the system can decide if the detections are to the side of the wheelchair or in front, and react appropriately. 

\subsection{Test System Description}
For this test, we use the same system as the one described in Section \ref{sec:hddSys}, with the addition of the Gazebo simulation of the ARTA powered wheelchair. We communicate the ARTA control signals to the Hololens, which checks the trajectory of the wheelchair and determines if a collision with a detected person is imminent. If the reactive control system decides a detected person is a collision risk, it reduces the velocity of the wheelchair in order to avoid a collision.

\subsection{Test Setup}

\subsubsection{Gazebo Simulation}
We use the Gazebo Simulation software to load a world built using the height map for the hallway outside the Personal Robotics Lab on the 10th floor. We then load the Unified Robot Description Format (URDF) model of ARTA into the world and run the ROS nodes controlling the powered wheelchair. We control the simulated wheelchair using the keyboard on the computer instead of a joystick.

\begin{figure}[ht]
	\begin{subfigure}[b]{.48\textwidth}
		\centering
		\includegraphics[width=1.0\linewidth]{img/chapter6_test/artamodel.jpg}
		\caption{URDF model of ARTA with joints that model the kinematics of the wheelchair.}
	\end{subfigure}%
	\hspace{\fill} 
	\begin{subfigure}[b]{.48\textwidth}
		\centering
		\includegraphics[width=1.0\linewidth]{img/chapter6_test/heightmap.jpg}
		\caption{An aerial view of the Lab Hallway on the 10th floor.}
	\end{subfigure}
	\vspace{-1\baselineskip}
	\begin{center}
		\caption{The Gazebo Simulation of the lab hallway outside the PRL. The blue mesh represents the range of the simulated laser scanner.}
		\label{fig:greenredrender}
	\end{center}
	\vspace{-1\baselineskip}
\end{figure}

\begin{figure}[ht!]
	\begin{subfigure}[b]{.48\textwidth}
		\centering
		\includegraphics[width=1.0\linewidth]{img/chapter6_test/gazeboBack.jpg}
		\caption{Gazebo Simulation of hallway from behind the model.}
	\end{subfigure}%
	\hspace{\fill} 
	\begin{subfigure}[b]{.48\textwidth}
		\centering
		\includegraphics[width=1.0\linewidth,height=41.5mm]{img/chapter6_test/realBack.jpg}
		\caption{Corresponding real world lab setup from behind.}
	\end{subfigure}

	\begin{subfigure}[b]{.48\textwidth}
		\centering
		\includegraphics[width=1.0\linewidth]{img/chapter6_test/gazeboFront.jpg}
		\caption{Gazebo Simulation of hallway from in front of the model.}
	\end{subfigure}%
	\hspace{\fill} 
	\begin{subfigure}[b]{.48\textwidth}
		\centering
		\includegraphics[width=1.0\linewidth,height=41.5mm]{img/chapter6_test/realFront.jpg}
		\caption{Target persons will stand on the markings in front of the test subject which are 1m apart.}
	\end{subfigure}
	\vspace{-1\baselineskip}
	\begin{center}
		\caption{We emulate the real world position of the Gazebo model position in the ICRS lab. We monitor the simulated velocity of the wheelchair as we vary the distance the target person is from the Hololens.}
		\label{fig:greenredrender}
	\end{center}
	\vspace{-2\baselineskip}
\end{figure}

\subsubsection{Real World Lab Setup}
We are trying to test how the velocity of the wheelchair changes when the system detects a collision risk. We model the movement of the wheelchair in the Gazebo simulation, but we need real human detections to test the reactive control system. We use the same setup used in Section \ref{sec:testSetup}, where we use the long hall between benches in the 5th floor ICRS lab to replicate the hallway outside the PRL lab.

\subsubsection{Monitoring ROS Topics}
We record the ROS topics publishing the linear and angular velocities of the simulated wheelchair using a ROS Bag. We monitor the following topics:

\begin{itemize}
	\item \code{/navigation/main\_js\_cmd\_vel}, the velocity controlled by joystick inputs.
	\item \code{/holo/cmd\_vel}, the reactive control velocity.
	\item \code{/arta/cmd\_vel}, the simulated velocity of the wheelchair.
\end{itemize}

\subsection{Test Procedure}
We asked the test subject to sit in a chair in the middle of the hallway wearing the Hololens. Similar to the experiment in Section \ref{sec:testSetup}, the target person stands at different distances to the test subject. Using the keyboard, we control the simulated wheelchair driving it forward down the simulated hallway. At each marked distance, we  record the joystick, reactive control and final velocities of the simulated wheelchair using the ROS bag for comparison. We test distances between $1$ and $4$ meters away, since the reactive control system only takes over for detections closer than $3$ meters. 

\paragraph{}After the distance tests, the test subjects were asked to look to the side as the simulated wheelchair moved forward. We asked the target person to stand to the side of the wheelchair to provide a human detection. This was done to test if the system can realize the detection is not in the trajectory of the wheelchair, and for the system to realize reactive control is not necessary.

\subsection{Results}
We present the results of this experiment in Figure \ref{fig:reactiveResults}. We note at $3$ meters that the reactive control system detects a collision risk and reacts appropriately by reducing the velocity of the wheelchair. We see in Figure \ref{fig:reactiveResults}.a that the output velocity of the reactive control system is below the ideal reactive control velocity of $0.5$m/s. Furthermore, the velocity oscillates between $0.3$m/s and $0.1$m/s, well below the ideal velocity. Figure \ref{fig:reactiveResults}.b shows a similar trend, whereby the output velocity of the reactive control system is $0$m/s instead of the ideal value of approximately $0.1$m/s in the ideal case. Similarly, Figure \ref{fig:reactiveResults}.c also shows a discrepancy. However, more significantly is that there are oscillations in the velocity caused by flickering detections. At a distance of 4 meters, the reactive control was not activated. As such, there is no difference between the joystick input and reactive control velocities, so we omit those results from the figure.

\paragraph{}With respect to the head turning results, we noted that the system was always able to recognize when the test subject was looking forward or to the side, and we noted no change to the joystick inputs when the user was not looking forward. This indicates that the reactive control system is not activated as intended.

\begin{figure}[]
	\begin{subfigure}[b]{1.0\textwidth}
		\centering
		\includegraphics[width=0.75\linewidth]{img/chapter6_test/reactive3.png}
		\caption{Upon detection, the reactive control reduces the velocity to around 0.3m/s. When the collision risk is gone, it returns to 1.0m/s.}
	\end{subfigure}%
	\hspace{\fill} 
	\begin{subfigure}[b]{1.0\textwidth}
		\centering
		\includegraphics[width=0.75\linewidth]{img/chapter6_test/reactive2.png}
		\caption{At a distance of 2m from the detection, the reactive control system prevents the wheelchair from moving at all.}
	\end{subfigure}
	\begin{subfigure}[b]{1.0\textwidth}
		\centering
		\includegraphics[width=0.75\linewidth]{img/chapter6_test/reactive1.png}
		\caption{The reactive control output velocity oscillates between 0 and 1 due to flickering detections. HDD fails to detect people if they are too close.}
	\end{subfigure}
	\vspace{-1\baselineskip}
	\begin{center}
		\caption{Reactive Control System velocity outputs at different distances.}
		\label{fig:reactiveResults}
	\end{center}
	\vspace{-2\baselineskip}
\end{figure}

\newpage

\subsection{Discussion} \label{sec:gazeboDiscussion}
Due to the issues with hologram placement accuracy discussed in Section \ref{sec:distanceDiscussion}, we can explain the discrepancy between the ideal reactive control velocity and the simulated reactive control. As was discussed in the previous section, the GameObjects and holograms are not always placed in the correct position or on the correct surface. As such, the distance between the PWU and the GameObject in the Unity world varies slightly compared to the real world measurement. When holograms are placed slightly closer to the PWU, the reactive control system responds by reducing the velocity more than the ideal case, as seen in Figure \ref{fig:reactiveResults}.

\paragraph{}We also wish to discuss the oscillations in the velocities set by the reactive control system. For the graph in Figure \ref{fig:reactiveResults}.a, the velocity oscillates within a small range below the ideal reactive control velocity. These small oscillations are most likely caused by flickering holograms. The holograms are ray cast onto the surface of the detected person, however, due to issues with hologram stability at distance, the placement of the hologram and GameObject oscillates between two values. This is rendered to the user as a hologram flickering between two positions on the detected person. As such, the distance between the target and the PWU changes, causing an oscillation in the velocity output of the reactive control system. 

\paragraph{}With regards to the larger oscillations seen in Figure \ref{fig:reactiveResults}.c, we believe that this is caused by the failure of the HDD system to detect a person standing that close to the PWU. However, we are unsure of the root cause of this problem. There is a possibility that this issue is caused by the YOLO detector being unable to detect a person when only their torso and head are visible. The more likely cause is that the OpenPose keypoint estimation fails due to it being unable to estimate the keypoints due to the person being too close. As part of our optimization to run both the tracker and pose estimator on the limited GPU memory, we reduced the network resolution of OpenPose, which results in less accurate key point estimations. The reduction in network resolution may also explain why the HDD system is unable to detect people further away, since the pose estimator fails for persons far away, as seen in Section \ref{sec:keypointEstimate}.

\paragraph{}Finally, we discuss the Hololens and ARTA frame alignment to determine if a detected person is in front of the wheelchair. From the tests, we noticed no change in the forward velocity of the simulated wheelchair. Furthermore, the cursor rendered on the holographic screens changes to the colour red when the PWU is not looking forward, and changes to green when the user is facing forward. This shows that the calibration process and alignment is correct, and that we can now test the full system.

\newpage

\section{Full System}
As a final test, we want to see the whole system in action with all three devices functioning and communicating with each other. We want to check whether the HDD system can detect people who are walking towards the PWU operating ARTA. We also want to see how the reactive control system responds to collision risks in a real world setting, and whether the the response is fast enough to compensate for people walking and the wheelchair moving. This test utilizes ARTA, the HDD system and the Hololens running the full Unity application.

\subsection{Test Setup}
This test was conducted in the hallway outside the Personal Robotics Lab on the 10th floor of the EEE building. The reason for this choice of test location is the pre-existing height maps for localisation and navigation that have been developed by the members of the PRL. Furthermore, this experiment was designed to test the real world application of the Gazebo Simulation test we conducted in Section \ref{sec:gazeboSimTest}.

\paragraph{}This experiment involves positioning ARTA at the end of the hallway. The test subject is sat in the wheelchair wearing the Hololens running the Unity application. Communication between the Hololens and the HDD system is established and checked to ensure the HDD system is receiving video frames and processing them. We then check the Hololens for detections by asking the test subject to confirm they can see holograms being rendered. We then ensure ARTA is communicating with the Hololens by trying to move the wheelchair without any detections and checking the \code{/holo/cmd\_vel/} ROS topic for messages.

\subsection{Test Procedure}
To test the effectiveness of the system, we designed three different scenarios:

\begin{enumerate}
	\item Target Away: ARTA and PWU driving forward in the same direction as the target person walking away down the hallway.
	\item Target Towards: ARTA and PWU driving forward as the target person walks towards the PWU.
	\item Looking Away: ARTA and PWU driving forward, PWU is looking to the side so that the target person is not in the FOV of the Hololens.
\end{enumerate}

We designed these scenarios with the system in mind. We want to test whether the Hololens will detect a person as a collision risk and translate it to a reactive control output that slows down the powered wheelchair, preventing a collision. We illustrate the positions and motions of the target person using the Gazebo simulation map in Figure \ref{fig:fullSystemTest} since it provides an aerial representation of the hallway.

\begin{figure}[ht]
	\centering
	\includegraphics[width=0.9\linewidth]{img/chapter6_test/fullSystem.png}
	\caption{The experimental setup used to test the full system. We ask the target person to walk down the hallway for Scenarios 1 \& 2. For Scenario 3, the target person stands still.}
	\label{fig:fullSystemTest}
\end{figure}

\paragraph{Target Away} In this scenario, we are testing the ability of detecting a person walking in the same direction as the wheelchair. We want to test the system response to the detection which we expect to be a reduction in the forward velocity of ARTA, allowing the robot to continue moving forward, albeit at a slower velocity. We also want to evaluate how the system responds to the person moving out of the frame.

\paragraph{Target Towards} This experiment was designed to test how the system responds to a collision risk that is approaching the wheelchair. The target person is asked to walk towards the wheelchair as the wheelchair moves forward. We are testing for the system to detect the person and rapidly decreasing the forward velocity of the wheelchair as the target person comes closer. We expect the system to stop moving completely when the target person is closer than 2m away.

\paragraph{Looking Away} We ask the test subject wearing the Hololens to drive ARTA in a forward direction while looking to the side. The target person is asked to stand still to the side 2m away from the start position of ARTA and the PWU. This scenario is designed to test how the system is able to recognize the detected person is not in the current trajectory of the wheelchair, and so the reactive control system does not need to modify the user joystick input commands to avoid a collision.

\paragraph{} During Scenarios 1 \& 2, we asked the Target Person to maintain a constant walking speed slightly faster than the top speed of 1.0m/s of the powered wheelchair. We also asked the Target Person to continue walking unless absolutely sure that a collision would occur, in which case we asked the Target Person to avoid the collision themselves.

\begin{figure}[ht]
	\begin{subfigure}[b]{.48\textwidth}
		\centering
		\includegraphics[width=0.5\linewidth]{img/chapter6_test/zihanBack.jpg}
		\caption{Scenario 1: Target person walking in the same direction as forward motion of ARTA.}
	\end{subfigure}%
	\hspace{\fill} 
	\begin{subfigure}[b]{.48\textwidth}
		\centering
		\includegraphics[width=0.5\linewidth]{img/chapter6_test/zihanForward.jpg}
		\caption{Scenario 2: Target person walking towards the PWU and ARTA.}
	\end{subfigure}
	\vspace{-1\baselineskip}
	\begin{center}
		\caption{The path taken by the Target Person along the hallway outside the PRL.}
		\label{fig:zihanBackForward}
	\end{center}
	\vspace{-2\baselineskip}
\end{figure}


\subsection{Results}
We measured the success of a scenario by whether human intervention was required to prevent a collision. We define human intervention as the act of:

\begin{itemize}
	\item The PWU stopping the forward joystick input manually or by navigating the wheelchair out of the way to prevent a collision.
	\item The Target Person having to move themselves off the assigned path to avoid a collision.  
\end{itemize}

\subsubsection{Target Away}
Through our observation of this scenario, we noticed several things:

\begin{itemize}
	\item System was unable to detect target person beyond 3-4 meters in front of the wheelchair.
	\item ARTA moved much slower than expected despite the target being between 2-3m in front of the device.
\end{itemize}

We define a detection as the placement of a hologram at the position of the detected object. The test subject was unable to see a hologram of the green arrow on the surface of the target person when the target person was standing further than 3-4 meters away. The PWU and ARTA retained a relatively smooth velocity behind the target person walking away and no collision with the target occurred. When the target person left the FOV of the PWU, the wheelchair returned to the top velocity of 1m/s.

\subsubsection{Target Towards}
In this scenario, the PWU and ARTA drove forwards as the target person walked towards the powered wheelchair. We noticed that as soon as the target person was approximately 3m away from the PWU, the velocity of ARTA reduced rapidly in a sudden jerking motion. As the target person got closer, we noted that ARTA oscillating between motion and being stationary, instead of remaining completely still. When the target person was very close to the PWU, ARTA began moving forward, indicating the reactive control system had failed to notice a detection and the system required human intervention. 

\subsubsection{Looking Away}
This test involved driving the wheelchair forward while the PWU looked at a target to the side. For this scenario, we noted no change in the velocity as the PWU detected the target person, which indicates that the reactive control system was not activated as expected.

\subsubsection{Summary of Scenarios}
To summarise the results of the scenarios, we want to highlight several points we noticed during the test. Firstly, we want to highlight that the system is able to determine whether a detection is in front of the powered wheelchair and in the way of the current trajectory. However, occasionally the system is unable to detect target persons in front of the wheelchair. This issue was observed when the target was either far away from the PWU or extremely close. Secondly, we note that the reactive control causes a sudden and much more rapid decrease in velocity than we intended. Finally, we want to highlight the issue the system faced when the target person was very close which caused Scenario 2 to require human intervention.

\subsection{Discussion}
We now discuss the issues highlighted in the previous section. We begin with the issue whereby the system fails to detect a target person standing very close to the PWU. As discussed in Section \ref{sec:distanceDiscussion} and Section \ref{sec:gazeboDiscussion}, this is most likely due to a failed detection by the HDD system for people too close to the PWU. This event is similar to the oscillation issue we faced in Section \ref{sec:gazeboDiscussion}, where for close detections, the velocity output of the reactive control system oscillated between 0m/s and 1m/s. 

\paragraph{} Furthermore, due to the variance in hologram placement distance, we can explain the rapid decrease in velocity output at 3m that occurred in Scenario 2. We believe that as the target person crossed the 3m distance and activated the reactive control system, the GameObject representing the person was placed slightly in front of the target. This incorrect placement caused the sudden drop in velocity. In addition, as the distance between the PWU and the target decreased, the ray casting of the GameObject had to correct itself multiple times, causing the distance between them to fluctuate. These fluctuation explain the rapid changes in velocity that caused ARTA to oscillate between movement and remaining stationary.

\paragraph{}As discussed in Section \ref{sec:gazeboDiscussion}, the lack of detections for people far away may be caused by incorrect pose estimations by the OpenPose network in the HDD system due to the low network resolution.

\subsection{Overall System Discussion}
By testing the full system, we were able to assess our implementation in a real world scenario. Although the real world scenario was simplified to only three controlled situations, we immediately noticed that there is a lot of room for improvement. First and foremost is in the HDD system, where persons far away were not recognized. From our tests, we established a possible limiting factor being the resolution of the OpenPose network, which may fail key point estimation for small figures in the distance. Secondly, in the Hololens Unity application, where we convert the image co-ordinate of the detection to a world co-ordinate. Here we rely on the library provided by Vulcan Technologies, which utilizes the camera parameters to form an un-projection matrix to obtain the world position of the pixel. Due to the delay from the streaming to a partner PC and processing through the HDD system, the world co-ordinate is slightly off and not representative of the world position of the person in the current frame. Finally, the rapid decrease in ARTA forward velocity when a detection enters the reactive control activation range. Due to the variance in hologram distance, we realized it was unsuitable to use a scaling function that reduced the speed of the wheelchair so aggressively. We discuss these issues further in the next chapter of the report.

\chapter{Evaluation}
This chapter summarises the capabilities of the augmented reality system as a product against the requirement capture, an evaluation of the techniques used to produce the system and the achievements of the project compared to the goals set in the interim report.

\paragraph{}In the interim report, we proposed the "Human Detection System," which would estimate the trajectories of people utilizing a moving object detector. Instead, the final product implements a Human Detection \& Direction system that relies on object detection and body pose estimation. The first and most challenging step of the project was developing the video stream of the front-facing camera on the Hololens. Due to the reliance of the HDD system on the visual input, this made the video stream a critical part of the project. As such, we considered multiple approaches to achieve a real-time stream with a relatively high frame rate. Using Unity, we were able to access the front-facing camera and stream the captured frames to a partner PC. From there, the HDD system can use the YOLO object detector to detect humans in the captured frames and discern the direction they are walking in using the OpenPose pose estimation network and object tracking.  As such, we have achieved the requirements set in Section \ref{sec:reqHDD}. 

\paragraph{}In spite of our achievements, we must be critical of the system we have implemented. Through our development, we found it was challenging to develop a video stream of the front-facing camera due to the limitations of the Unity game engine. We discuss the issues of compression in Section \ref{sec:framerate}, and how it limits the user experience to 5 FPS on the holographic lenses of the Hololens. Despite discussing potential solutions with the members of the PRL and open source contributors online, we were unable to find a realistic solution for faster image compression that could be run outside of the Unity main thread. As such, we had to continue with the project, knowing that there was a limitation on the user experience in terms of holographic stability. Furthermore, due to the limitations of the front facing camera (which captures at 30 FPS), image compression and transfer over the network, we can only achieve a modest 10 FPS on the partner PC. 

\paragraph{}With regards to the person detector in the HDD system, we were able to improve the detections of partially occluded persons and smaller figures at a distance compared to the pre-trained model. We verified this through our visual testing on the MOT dataset and our recorded videos around Imperial College London. On the other hand, we did not end up using the object tracking capabilities of Deep SORT to its full potential. We initially proposed the use of the tracker to determine the directions using only a 2D image by tracking the motions of objects across frames. However, we found this task difficult when the detected person was walking towards or away from the camera, as discussed in Section \ref{sec:objecTrackingDirection}. However, instead of removing the tracker, we kept it in the system for potential future work to increase the hologram stability and better direction estimation.

\paragraph{}Furthermore, the decision to use body pose estimation to determine the direction a person is walking in was an adventurous one. A much simpler option would have been to rely on object tracking and using the depth camera on the Hololens to see how the position of the object changed in the real world. This would have reduced the GPU memory usage, the complexity of the HDD system, and the amount of work required to get the network to run on ROS. In hindsight, we realize it was misguided to rely solely on computer vision techniques to determine the human direction and that a much safer option would have been to utilize the sensors for depth perception available on the Hololens and ARTA.

\paragraph{} Also, in the interim report, we proposed the use of the Pupil Labs eye tracker as a form of wheelchair control. However, streaming the captured video frames caused a limitation in the frame rate of the user experience, and as such, we were unable to implement this goal. We discuss the full reasoning for this in Section \ref{sec:eyeTracker}. We also proposed the development of Hologram warnings using the eye tracker to alert the user on a potential collision. In the final implementation, we have developed hologram visualizations of the directions detected people are walking in which act as a warning for potential collisions. We achieve this through the detections produced by the HDD system and create a map of the detections by utilizing the spatial mapping. Since one of the goals set in the interim report was to control the powered wheelchair using the augmented reality headset, we implemented the reactive control system across ARTA and the Hololens. However, from our testing, we found that there are several issues with the reactive control system, and most of them stem from the fact that we are using a single video input as our source of decision making.

\paragraph{}From our results, we realize that the combination of a low frame rate from the video stream and the network delay during transfer between the Hololens and partner PC results in latency between the processed frame and current view of the PWU. This latency also results in incorrect projections of image pixels to world coordinate space. As discussed in Section \ref{sec:fullSystemTest}, this causes holograms to jitter in position, as well as affecting the output of the reactive control system velocity. To avoid the issue of inaccurate hologram placement in the world, a solution would have been to use Spatial Anchors and placing holograms relative to the anchors. This would also increase the accuracy of the hologram in the real world relative to the virtual world rendered by the Hololens. However, we did not use spatial anchors because one of the ideas for this system is for PWU to be able to wear it in the streets, in unmapped places where spatial anchors are not already placed. 

\paragraph{}Finally, we discuss the reactive control system, which we implement on the Hololens and ARTA. The system relies solely on the mapping of detected people created by the Unity application. However, we found that this results in a system that can be unstable, depending on the accuracy of the detections. From our tests, we also noticed that the lighting of the room affected the systems ability to detect people, due to the resolution of the front-facing camera. As such, a better approach for collision avoidance would have been to utilize the PRL developed obstacle avoidance ROS packages built for ARTA. These libraries would have made the development of the reactive control system much simpler. Furthermore, from our tests, we also noticed that the wheelchair would reduce its velocity rapidly when detecting an object moving towards it. This is due to the choice of scaling function, which are much too aggressive for real-world use. Instead, a smoother scaling function that reduces the velocity slowly is more suitable for a powered wheelchair, resulting in a smoother ride.

\chapter{Conclusion and Further Work}

\section{Conclusion}
Throughout this report, we have discussed the implementation of an augmented reality system to aid PWUs in navigating areas with people as potential collision risks. We outlined the requirements for the system to handle this scenario in the Requirement Capture section of the report. From our tests, we can show that we have implemented a working proof of concept, using the Microsoft Hololens as the augmented reality device, and front-facing camera as the visual input for people detection and analysis.

\paragraph{}One of the contributions of this project is the development of a Unity application that streams the front-facing camera of the Hololens to a partner PC. The ability to process the frames on a computer with a GPU opens up a whole avenue of research for computer vision techniques using the Hololens. This report explores the use of these techniques in the Human Detection \& Direction system, which uses the YOLO object detector, Deep SORT tracker, and OpenPose body pose estimation networks. To make this accessible to future researchers, we make available the Unity framework, which implements the video streaming across ROS topics. However, this report also highlights the limitations of using the Unity engine for video streaming, and suggest that any future researchers should develop a UWP application that accesses the Hololens video stream, to make use of the multi-threading capabilities of the device.

\paragraph{}Secondly, we contribute a version of Darknet that has been modified to run as a ROS node, allowing seamless integration with other ROS packages. We achieved this by wrapping the core Darknet libraries using Python and providing an interface between the neural network framework and ROS topics to pass messages containing the images and bounding box detections. In addition to this, we have also trained the YOLO object detector using the CrowdHuman dataset. We provide the pre-trained weights that can be run on the ROS Darknet framework. We also explored the use of body pose estimation as a method of determining the direction a detected individual is walking in. We have developed Python wrappers around the network to allow it to accept ROS topic messages as input and a way for it to output its keypoint estimations.  

\paragraph{}Thirdly, in addition to the video streaming capabilities of the Unity application, we have also developed an augmented reality experience for PWU which visualizes the direction people are walking in, based on the results of the HDD system. These holograms are rendered by the Hololens and are used as visual aids by the PWU to avoid collisions. Due to the frame rate limitations of the application, the holographic visualizations are not completely stable but provide a reasonable indicator of direction to the PWU.

\paragraph{}Finally, we have developed a simple collision avoidance system that utilizes the spatial mapping capabilities of the Hololens to determine the distance between the PWU and the detected people. The reactive control system monitors the input velocity commands sent by the PWU using the joystick on ARTA and selectively controls the final velocity of the wheelchair when it detects a collision.

\section{Future Work}

\subsection{Utilizing ARTA Sensors}
To achieve a more accurate mapping of the surroundings, the system should not rely on a single sensor input in the form of the front-facing camera. Instead, the visual input and positions of the object obtained from the front-facing camera should be used in conjunction with other sensors, such as the laser scanners attached to the base of ARTA. Furthermore, the PRL has already developed ROS packages that implement this on ARTA. The future work would involve comparing the positions of the detected persons in the Hololens frame and ARTA frame to estimate a better world position.

\subsection{GameObject Tracking in Unity}
In the current Unity application, we create a new GameObject representing a detected person and delete it every frame. We do this since we determine the direction a person is walking in from the output of the OpenPose network. As such, the tracker ID is not used beyond visualization.

\paragraph{} An alternative approach would be to create a single GameObject for each tracking ID. These GameObjects are then compared across frames, and if the tracker ID exists in the new frame, the previously instantiated GameObject is rendered. When the tracker ID is not in the frame, we set the mesh of the hologram to transparent, and the hologram is not visible to the PWU. This would allow the Unity application to keep track of the positions of the detected objects across frames, and reduce the error in the accuracy of hologram placement. This would also mean that the Hololens was aware of detected persons, even when the user turns their head to the side and the objects are no longer in the FOV of the front-facing camera.

\subsection{Advanced Reactive Control}
Rather than have the reactive control system reduce the speed of ARTA when it detects a collision risk, a more advanced technique would be a system that takes over and navigates the wheelchair out of the way of the detected person. This system would be implemented using either the additional ARTA sensors or GameObject tracking we mentioned in the previous sections as a way of keeping a better mapping of the surroundings of the Hololens. We could then use the object avoidance libraries developed by the PRL to control ARTA when it detects objects, and calculate the best path that avoids a collision.

\subsection{Video Streaming}
The major limitation of the Unity Game engine is the fact that image compression can only be done in the main thread of the application. This limits the frame rate the scene can be displayed at, resulting in latency when rendering holograms in the surroundings. This was why we chose to instantiate and delete holograms for each detection instead of having them persist across frames. 

\paragraph{}To solve this problem, one would have to develop a way of compressing the images outside of the main thread. This is a difficult problem to solve since Unity does not allow access to textures and GameObjects outside of the main thread due to thread safety issues. As such, an alternative approach would be to somehow access the video streams outside of Unity, perhaps using the HololensForCV library provided by Microsoft, to stream the front facing camera to a partner PC. The partner PC would then stream the captured images back to the Hololens and the detections, into the Unity application. This, however, would involve writing a UWP application that could run in the background of the Hololens, as well as using Windows networking protocols to stream the video data from the Hololens.


\bibliographystyle{unsrt}
\bibliography{fyp_report.bib}

\begin{appendices}
\chapter{Software}
\section{Developed Software}
All code developed in this project is available in their respective GitHub repositories. We have provided instructions in the \code{README.md} file of each repository on how to setup the code and how to run it.

\subsection{Partner PC}
The following applications were developed to run on Ubuntu 16.04 with 8GB DDR3 RAM and a GTX 1050 Ti. ROS Kinetic is also required, as well as Python 2, gcc and g++. For a more complete listing, please refer to the medium 

\subsubsection{Darknet CrowdHuman}
\paragraph{Source:} \code{https://github.com/alaksana96/darknet-crowdhuman}

\paragraph{}This repository contains the Darknet version used in this project. We also provide the code required to convert the CrowdHuman dataset to the Darknet format for training the YOLO detector. This repository is also used as a submodule in the \code{fyp\_yolo} repository.

\subsubsection{ROS Darknet/YOLO}
\paragraph{Source:} \code{https://github.com/alaksana96/fyp\_yolo}

\paragraph{}The ROS wrapper for the Darknet neural network framework. This repository relies on the Darknet CrowdHuman repository, and clones it as a submodule. When cloning this repository, please remember to clone it recursively using the \code{--recurse-submodules} flag

\subsubsection{Yet Another CrowdHuman Tracker}
\paragraph{Source:} \code{https://github.com/alaksana96/fyp\_yact}

\paragraph{Dependencies:}
\begin{itemize}
	\item \code{https://github.com/CMU-Perceptual-Computing-Lab/openpose}
	\item \code{https://github.com/alaksana96/deep\_sort}
\end{itemize}

\paragraph{}YACHT depends on the OpenPose network and a modified DeepSORT implementation. Please refer to the OpenPose repository for installation instructions. Furthermore, please clone the YACHT repository recursively using the \code{--recurse-submodules} flag.

\subsection{Hololens}
The following applications were developed using Unity for the Unity Engine running on the Microsoft Hololens.

\subsubsection{Unity Application}
\paragraph{Source:} \code{https://github.com/alaksana96/fyp\_hololens\_unity}

\paragraph{}This application requires Unity 2018.1.6f1 and Visual Studio 2017 Community edition to build and deploy to the Hololens.

\section{Personal Robotics Lab Software}
For the software developed by the Personal Robotics Lab, please email a member of the lab for access to the repositories:

\begin{itemize}
	\item \code{https://github.com/ImperialCollegeLondon/arta}
	\item \code{https://github.com/ImperialCollegeLondon/prl\_localisation}
	\item \code{https://github.com/alaksana96/prl\_navigation}
\end{itemize}

We have modified the \code{prl\_navigation} repository to work with our reactive control system, please email contact us on GitHub for access to the repository.

\end{appendices}

\end{document}